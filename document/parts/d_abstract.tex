\chapter*{Resumen}

La gestión del conocimiento personal (PKM) enfrenta el desafío de integrar eficientemente nueva información. Esta tesis aborda este problema mediante el desarrollo de un sistema que automatiza la construcción incremental de bases de conocimiento semiestructuradas utilizando Grandes Modelos de Lenguaje (LLM). Se presenta \textbf{lImporter}, un plugin para Obsidian.md implementado como un agente autónomo. La arquitectura del agente se basa en el paradigma \textit{ReAct} (Razonamiento y Acción), permitiéndole planificar y ejecutar tareas complejas. Utiliza \textit{function calling} para interactuar de forma fiable con la bóveda de notas del usuario y un sistema de optimización de contexto para analizar eficientemente el conocimiento existente y encontrar conexiones relevantes.

La eficacia y versatilidad del sistema se evaluaron a través de una serie de experimentos cualitativos. Estos demostraron la capacidad del agente para: (1) construir grafos de conocimiento estructurados a partir de texto no estructurado; (2) mantener y expandir una base de conocimiento existente, creando enlaces contextualmente apropiados; (3) adaptarse a flujos de trabajo de productividad como P.A.R.A.; y (4) utilizar la base de notas como una memoria externa para resolver tareas iterativas, sugiriendo su potencial como herramienta computacionalmente universal. El trabajo concluye que la integración de agentes LLM en sistemas de PKM transforma estas herramientas de repositorios pasivos a colaboradores cognitivos activos, capaces de estructurar, conectar y generar nuevo conocimiento de manera autónoma.

\vspace{1cm}
\textbf{Palabras clave:} Gestión del Conocimiento Personal, Modelos de Lenguaje, Obsidian, Agentes Autónomos, ReAct, Grafos de Conocimiento.

\chapter*{Abstract}

Personal Knowledge Management (PKM) faces the challenge of efficiently integrating new information. This thesis addresses this problem by developing a system that automates the incremental construction of semi-structured knowledge bases using Large Language Models (LLMs). We introduce \textbf{lImporter}, an Obsidian.md plugin implemented as an autonomous agent. The agent's architecture is based on the \textit{ReAct} (Reasoning and Acting) paradigm, enabling it to plan and execute complex tasks. It leverages \textit{function calling} for reliable interaction with the user's note vault and employs a context optimization system to efficiently analyze existing knowledge and find relevant connections.

The system's effectiveness and versatility were evaluated through a series of qualitative experiments. These demonstrated the agent's ability to: (1) build structured knowledge graphs from unstructured text; (2) maintain and expand an existing knowledge base by creating contextually appropriate links; (3) adapt to productivity workflows such as P.A.R.A.; and (4) use the note vault as external memory to solve iterative tasks, hinting at its potential as a computationally universal tool. This work concludes that integrating LLM agents into PKM systems transforms these tools from passive repositories into active cognitive collaborators, capable of autonomously structuring, connecting, and generating new knowledge.

\vspace{1cm}
\textbf{Keywords:} Personal Knowledge Management, Large Language Models, Obsidian, Autonomous Agents, ReAct, Knowledge Graphs.